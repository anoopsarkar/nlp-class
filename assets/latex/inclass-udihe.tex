
\documentclass[11pt]{article}
\usepackage{txfonts}
\usepackage{graphicx}
\usepackage{mygb4e}
\usepackage{tipa}
\usepackage{solution}

%\hidesolutions
\setlength\oddsidemargin{0.01in}
\setlength\topmargin{-1in}
\setlength\textwidth{6.9in}
\setlength\textheight{9.5in} 

\newcommand{\nl}{\mbox{$\langle cr \rangle$}}

\raggedright

\begin{document}

\begin{center}
{\Large Udihe}\\
{\small B. Iomdin -- modified by Anoop Sarkar {\tt anoop@cs.sfu.ca}}
\end{center}

The following is a small parallel text (the same text in two different
languages). The 1st column contains phrases in the Southern (Bikin) dialect
of the Udihe language. The 2nd column contains the
English equivalent. 

\bigskip

\begin{tabular}{ll}
\textipa{b"ata z\"a:Nini} & the boy's money \\
\textipa{si bogdoloi} & thy shoulder \\
\textipa{ja: xabani} & the cow's udder \\
\textipa{su z\"a:Niu} & your money \\
\textipa{dili tekpuni} & the skin of the head \\
\textipa{si ja:Ni:} & thy cow \\
\textipa{bi mo:Ni:} & my tree \\
\textipa{aziga bugdini} & the girl's leg \\
\textipa{bi nakta diliNi:} & my boar head \\
\textipa{nakta igini} & the boar's tail \\
\textipa{si b"ataNi: bogdoloni} & thy son's shoulder \\
\textipa{teNku bugdini} & the leg of the stool \\
\textipa{su ja: wo:Niu} & your cow thigh \\
\textipa{bi wo:i} & my thigh
\end{tabular}

\bigskip

\textipa{N}, \textipa{"} are consonants, \textipa{\"a} is a vowel. The
\textipa{:} indicates length of preceding vowel. The archaic {\it thy}
is used to indicate singular and {\it your} is used to indicate plural.

\bigskip

\begin{soln}

\smallskip

Consider the English phrase {\it X's Y} or {\it Y of the X}. The following table
summarizes how this phrase has to be structured in Udihe:

\smallskip

\begin{tabular}{|l|p{1.4in}|p{3in}|}
\hline
X (possessor) & Udihe phrase for {\it X's Y} or {\it Y of the X} & examples \\
\hline
singular \& I/you/my/thy & X Y-(\textipa{Ni})-\textipa{i} & 
  \textipa{bi wo:\underline{i}}, \textipa{bi mo:\underline{Ni:}},
  \textipa{bi nakta dili\underline{Ni:}},
  \textipa{si bogdolo\underline{i}}, \textipa{si ja:\underline{Ni:}},
   \textipa{si b"ata\underline{Ni:} bogdoloni} \\
singular (all other cases) & X Y-(\textipa{Ni})-\textipa{ni} &
  \textipa{ja: xaba\underline{ni}}, \textipa{dili tekpu\underline{ni}}, 
  \textipa{b"ata z\"a:\underline{Nini}}, 
  \textipa{si b"ataNi: bogdolo\underline{ni}} \\
plural & X Y-(\textipa{Ni})-\textipa{u} & 
  \textipa{su z\"a:\underline{Niu}}, \textipa{su ja: wo:\underline{Niu}} \\
\hline
\end{tabular}

\smallskip

Notice that \textipa{Ni} occurs exactly in those cases when, in the phrase
{\it X's Y}, the {\it Y} is not in a part-whole relationship with respect
to {\it X}. For example, \textipa{bi wo:i} (my thigh) is in a part-whole
relationship, while \textipa{bi mo:Ni:} (my tree) is not in a part-whole
relationship. Also, \textipa{Ni}+\textipa{i} becomes \textipa{Ni:} since
the vowel is simply lengthened.

\smallskip

In the case where the possessor is itself a possession phrase
e.g. {\it \fbox{thy son}'s shoulder}, each possessee gets the 
appropriate suffix, e.g. \textipa{si b"ata\underline{Ni:} bogdolo\underline{ni}}.
And in the case where the possessee is itself a possession phrase, 
e.g.  {\it my \fbox{boar head}}, only the actual part possessed is
marked, e.g. \textipa{bi nakta dili\underline{Ni:}}. How would you say
{\it the skin of the head of the cow} in Udihe?

\bigskip

Consider the pronouns observed in the parallel text:

\smallskip

\begin{tabular}{|c|c|}
\hline
singular & plural \\
\hline
\textipa{bi}/I & (\textipa{bu})/our \\
\textipa{si}/you & \textipa{su}/your \\
\hline
\end{tabular}

\smallskip

Note that the pronoun {\it our} does not occur in the text, but by
analogy to {\it you} vs. {\it your} we can conjecture that the plural
of {\it I} which is {\it our} in English, will be \textipa{bu} in Udihe.

\bigskip

Another missing form we can construct using analogy is the word for
{\it daughter} which is not observed, but we do observe the words for
{\it boy} and {\it son}:

\smallskip

\begin{tabular}{|c|c|}
\hline
\textipa{b"ata}/boy & \textipa{b"ata}/son \\
\textipa{aziga}/girl & (\textipa{aziga})/daughter \\
\hline
\end{tabular}

\end{soln}

\begin{exe}

\ex Translate into English:

\begin{xlist}
{\ex \textipa{su b"ataNiu z\"a:Nini}

\begin{soln}
your son's money
\end{soln}
}

{\ex \textipa{si teNku bugdiNi:}

\begin{soln}
thy stool leg
\end{soln}
}

{\ex 
\textipa{si teNkuNi: bugdini}

\begin{soln}
thy stool's leg
\end{soln}
}

\end{xlist}

\ex Translate into Udihe:

\begin{xlist}
{\ex the boy's thigh

\begin{soln}
\textipa{b"ata wo:ni}
\end{soln}
}

{\ex our boar

\begin{soln}
\textipa{bu naktaNiu}
\end{soln}
}

{\ex my daughter's tree

\begin{soln}
\textipa{bi azigaNi: mo:Nini}
\end{soln}
}

\end{xlist}

\end{exe}

Udihe speakers mostly live in the Siberian far east, 
and the language is classified as belonging to the Tungus-Manchu language
family. There are roughly 100 people who still speak this language. The language
is almost extinct. Other than the parallel text given above, you do not need any knowledge about the language
and its speakers to answer the questions, but if you are curious, here are
some web pages on the Udihe language:

{\small
\begin{verbatim}
http://www.ethnologue.com/show_language.asp?code=ude
http://en.wikipedia.org/wiki/Udege_language
\end{verbatim}
}

\end{document}

